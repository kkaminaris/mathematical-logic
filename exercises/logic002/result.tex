Συμπερασματικά, στην περίπτωση που είναι κοινή γνώση πως ο παίκτης C είναι τυφλός, τότε το αν αυτός θα μπορέσει να μάθει το χρώμα του εξαρτάται από τον τρόπο που αποκρίνονται οι παίκτες. Πιο συγκεκριμένα, αν οι παίκτες αποκρίνονται διαδοχικά (πρώτα ο A, μετά ο B, μετά ο C), τότε ο παίκτης C θα μπορέσει να απαντήσει θετικά με την πρώτη ερώτηση του βασιλιά. Αν όμως οι παίκτες αποκρίνονται ταυτόχρονα, τότε ο παίκτης C μπορεί να μη μπορέσει να αποκριθεί με την πρώτη ερώτηση του βασιλιά, και να χρειαστεί άλλες δύο ερωτήσεις για να απαντήσει θετικά.