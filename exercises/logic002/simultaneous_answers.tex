Στην ανάλυση που ακολουθεί έχει γίνει η σημαντική παραδοχή πως οι απαντήσεις των παικτών είναι ταυτόχρονες (και όχι με τη σειρά). Επίσης έχει γίνει η (λιγότερο σημαντική) παραδοχή πως τα καπέλα είναι 3 λευκά και 2 μαύρα αντί για τα 3 κόκκινα και 2 λευκά της εκφώνησης. Θα χρησιμοποιηθούν οι ατομικές προτάσεις $AisWh$, $BisWh$ και $CisWh$, με την ευκόλως εννοούμενη ερμηνεία τους.

\begin{table}[h!]
\centering
\begin{tabular}{ c|c c c }
  & A & B & C \\
 \hline
 s1 & W & W & W \\ 
 \hline
 s2 & B & W & W \\
 \hline
 s3 & W & B & W \\ 
 \hline
 s4 & W & W & B \\ 
 \hline
 s5 & B & B & W \\ 
 \hline
 s6 & B & W & B \\ 
 \hline
 s7 & W & B & B \\ 
\end{tabular}
\caption{Πιθανοί κόσμοι}
\label{worlds}
\end{table}

\begin{figure}[h!]
\centering
\begin{tikzpicture}[modal, node distance=3cm]

    \node[world] (1) [label=above:$W W W$] {$s1$};
    \node[world] (2) [label=left:$B W W$,below left=of 1] {$s2$};
    \node[world] (3) [label=right:$W B W$] at (2 -| 1) {$s3$};
    \node[world] (4) [label=right:$W W B$,below right=of 1] {$s4$};
    \node[world] (5) [label=below:$B B W$,below left=of 2] {$s5$};
    \node[world] (6) [label=below:$B W B$] at (5 -| 3) {$s6$};
    \node[world] (7) [label=below:$W B B$,below right=of 4] {$s7$};
    
    \path[-]
        (1) edge[color=cyan] node[] {$B$} (3)
        (2) edge[] node[] {$A$} (1)
            edge[color=cyan] node[] {$B$} (5)
        (3) edge[] node[] {$A$} (5)
        (4) edge[] node[] {$A$} (6)
            edge[color=cyan] node[] {$B$} (7)
    ;        
\end{tikzpicture}
\caption{M1}
\label{M1}
\end{figure}

Θα ασχοληθούμε με το ερώτημα β'. Η λίστα με όλους τους πιθανούς κόσμους φαίνεται στον πίνακα \ref{worlds}. Το αντίστοιχο διάγραμμα Kripke φαίνεται στο σχήμα \ref{M1}. Εφόσον είναι κοινή γνώση πως ο παίκτης C είναι τυφλός, όλοι οι πιθανοί κόσμοι είναι ισοδύναμοι για τον παίκτη C. Επομένως υπάρχουν ακμές για τον C από κάθε κόσμο σε κάθε άλλο κόσμο. Οι ακμές αυτές δεν αναπαρίστανται για λόγους ευκρίνειας. Παρατηρούμε πως $(K_A\ AisWh)$ αληθεύει μόνο στην κατάσταση $s7$, ενώ $(K_B\ BisWh)$ ισχύει μόνο στην κατάσταση $s6$.

Θα διακρίνουμε τώρα δύο περιπτώσεις. Αν στην πρώτη ερώτηση του βασιλιά είτε ο A, είτε ο B απαντήσουν πως ξέρουν το χρώμα του καπέλου τους, τότε βρισκόμαστε είτε στην κατάσταση $s7$, είτε στην κατάσταση $s6$ αντίστοιχα. Παρατηρούμε τώρα πως ο παίκτης C έχει μαύρο καπέλο και στις δύο αυτές καταστάσεις. Άρα όλοι οι παίκτες ξέρουν το χρώμα τους μετά τη δεύτερη ερώτηση του βασιλιά. Αντιθέτως αν όλοι απαντήσουν αρνητικά στην πρώτη ερώτηση, τότε το διάγραμμα Kripke αλλάζει όπως φαίνεται στο σχήμα \ref{M2} (Οι ακμές του παίκτη C παραλείπονται για λόγους ευκρίνειας).

\begin{figure}[h!]
\centering
\begin{tikzpicture}[modal, node distance=3cm]

    \node[world] (1) [label=right:$W W W$] {$s1$};
    \node[world] (2) [label=left:$B W W$,left=of 1] {$s2$};
    \node[world] (3) [label=right:$W B W$,below=of 1] {$s3$};
    \node[world] (4) [label=right:$W W B$,above right=of 3] {$s4$};
    \node[world] (5) [label=left:$B B W$,below=of 2] {$s5$};
    
    \path[-]
        (1) edge[color=cyan] node[] {$B$} (3)
        (2) edge[] node[] {$A$} (1)
            edge[color=cyan] node[] {$B$} (5)
        (3) edge[] node[] {$A$} (5)
    ;        
\end{tikzpicture}
\caption{M2}
\label{M2}
\end{figure}

Παρατηρούμε τώρα ότι $(K_A\ AisWh \land K_B\ BisWh)$ αληθεύει μόνο στην κατάσταση $s4$. Θα διακρίνουμε όπως πριν δύο περιπτώσεις. Αν στη δεύτερη ερώτηση ο A και ο B απαντήσουν θετικά, τότε ο παίκτης C μαθαίνει πως φοράει μαύρο καπέλο και μπορεί να απαντήσει θετικά στην τρίτη ερώτηση. Αντιθέτως αν όλοι απαντήσουν αρνητικά στη δεύτερη ερώτηση, τότε το διάγραμμα Kripke αλλάζει όπως φαίνεται στο σχήμα \ref{M3}. Παρατηρούμε όμως πως $(CisWh)$ αληθεύει σε όλες τις πιθανές καταστάσεις του διαγράμματος, κι έτσι ο παίκτης C θα απαντήσει θετικά στην τρίτη ερώτηση. Ταυτόχρονα όμως, επειδή $(CisWh)$ αληθεύει σε όλες τις πιθανές καταστάσεις, ισχύει ότι $(C\ CisWh)$. Σε αυτή την περίπτωση οι παίκτες A και B δε θα μάθουν ποτέ το χρώμα του καπέλου τους, όσες φορές κι αν ρωτήσει ο βασιλιάς. Το διάγραμμα Kripke μένει στάσιμο από εδώ και στο εξής γιατί κανένας παίκτης δεν αποκτά καινούργια πληροφορία.

\begin{figure}[h!]
\centering
\begin{tikzpicture}[modal, node distance=3cm]

    \node[world] (1) [label=right:$W W W$] {$s1$};
    \node[world] (2) [label=left:$B W W$,left=of 1] {$s2$};
    \node[world] (3) [label=right:$W B W$,below=of 1] {$s3$};
    \node[world] (5) [label=left:$B B W$,below=of 2] {$s5$};
    
    \path[-]
        (1) edge[color=cyan] node[left] {$B$} (3)
            edge[bend left, color=purple] node[] {$C$} (3)
            edge[bend right, color=purple] node[] {$C$} (5)
        (2) edge[] node[] {$A$} (1)
            edge[color=cyan] node[] {$B$} (5)
            edge[bend left, color=purple] node[] {$C$} (1)
            edge[bend left, color=purple] node[] {$C$} (3)
            edge[bend right, color=purple] node[] {$C$} (5)
        (3) edge[] node[above] {$A$} (5)
            edge[bend left, color=purple] node[] {$C$} (5)
    ;        
\end{tikzpicture}
\caption{M3}
\label{M3}
\end{figure}